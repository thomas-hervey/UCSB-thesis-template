\begin{abstract}
Many searches for geospatial data on online web portals fail or are abandoned. One reason is that search results are not relevant enough. Unfortunately, relevance, and in particular geographic relevance, are not well defined or agreed upon in the geographic information retrieval (GIR) community \cite{Reichenbacher2011}. To help reduce search failure and abandonment, there needs to be a more encompassing understanding of relevance in search and an understanding of how users search for geospatial data \cite{Purves2018}. One promising approach is to better capture and reason with the explicit and implicit concepts that users use when they search. In this proposal, I propose methods to answer the question \emph{can the concepts (e.g., granularity) that people use implicitly and explicitly when searching for geospatial data improve ranking in GIR systems to yield more relevant search results?} 
Specifically, I introduce five research tasks (herein called studies) to 1) survey the state of faced search on geospatial data portals, 2) analyze query logs for user search behavior, search refinement, and concepts that users use that may indicate relevance, 3) see how location as a dimension affects search behavior, refinement, and concepts used, 4) apply these findings to a real search ranking function, and 5) evaluate if the application improves search effectiveness and reduces search failure.
\end{abstract}
