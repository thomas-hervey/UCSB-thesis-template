% Glossary

\newglossaryentry{application_programming_interface}
{
    name=Application Programming Interface,
    text=application programming interface,
    description={A programmatic medium for interacting with a system, typically for acquiring data}
}


\newglossaryentry{bag_of_words}
{
    name=Bag of Words,
    text=bag of words,
    description=A natural language processing model for representing text as a bag of individual terms regardless of grammar or order
}

\newglossaryentry{click_deviation}
{
    name=Click Deviation,
    text=click deviation,
    description={A pattern where a user deviates from clicking on the top SERP result. For example, a deviation where a user selects the third result in an ordered list implies that they find the third result more relevant than the second or first result. Typically, deviation is used as indirect relevance feedback and used to improve ranking.}
}

\newglossaryentry{Dirichlet_Multinomial_Mixture}
{
    name=Dirichlet Multinomial Mixture for short text clustering,
    text=Dirichlet Multinomial Mixture for short text clustering,
    description={A text clustering algorithm that specializes in short text. Abbreviated as GSDDM}
}

\newglossaryentry{discounted_cumulative_gain}
{
    name=Discounted Cumulative Gain,
    text=discounted cumulative gain,
    description=A quality rank measure for capturing non-binary notions of relevance
}

\newglossaryentry{document}
{
    name=Document,
    text=document,
    description=The units that are decided to build a retrieval system over \cite{Manning2008}
}

\newglossaryentry{gazetteer}
{
    name=Gazetteer,
    text=gazetteer,
    description={A geographical index typically containing a place name, a georeference, and a type}
}

\newglossaryentry{geographic_information_needs}
{
    name=Geographic Information Needs,
    text=geographic information needs,
    description=A desire to obtain geographic information to satisfy a conscious or unconscious geographically inclined or constrained need or goal
}

\newglossaryentry{geographic_information_retrieval}
{
    name=Geographic Information Retrieval,
    text=geographic information retrieval,
    description=The information retrieval ecosystem for handling and serving geographic information [see information retrieval]
}

\newglossaryentry{geographic_information_system}
{
    name=Geographic Information System,
    text=geographic information system,
    description={A computer-based system designed to capture, manage, manipulate, analyze, and visualizes spatial or geographic data}
}

\newglossaryentry{geographic_relevance}
{
    name=Geographic Relevance,
    text=geographic relevance,
    description={he quality of an entity in geographic space or its representation in an information system, i.e. an object, document, or image \cite{Reichenbacher2011}}
}

\newglossaryentry{geoparser}
{
    name=Geoparser,
    text=geoparser,
    description={A tool that processes text and identifies geographic components of text, typically including modules for geotagging (identifying geographic elements) and georesolution (matching elements to proper information objects and/or real world objects)}
}

\newglossaryentry{georeference}
{
    name=Georeference,
    text=georeference,
    description=Assigning a reference value to an information object based on a reference system such as a coordinate reference system
}

\newglossaryentry{geospatial_datasets}
{
    name=Geospatial Dataset,
    text=geospatial datasets,
    description=A collection of geospatial data features such as georeferenced points with unifying metadata
}

\newglossaryentry{information_object}
{
    name=Information Object,
    text=information object,
    description=An object-oriented abstract data model used to specify information about real-world objects
}

\newglossaryentry{information_need}
{
    name=Information Need,
    text=information need,
    description=A desire to obtain information to satisfy a conscious or unconscious need or goal
}

\newglossaryentry{information_retrieval}
{
    name=Information Retrieval,
    text=information retrieval,
    description={The representation, storage, organization of, and access to information items such as documents, Web pages, online catalogs, structured and semi-structured records, multimedia objects. The representation and organization of the information items should be such as to provide the user with easy access to information of their interest \cite{Baeza-Yates1999}}
}

\newglossaryentry{information_seeking}
{
    name=Information Seeking,
    text=information seeking,
    description=A searching behavior driven by an information need where the searcher develops strategies to obtain information
}

\newglossaryentry{Latent_Dirichlet_Allocation}
{
    name=Latent Dirichlet Allocation,
    text=Latent Dirichlet Allocation,
    description=A natural language processing generative probabilistic model often used to classify and model topics
}

\newglossaryentry{learn_to_rank}
{
    name=Learn-to-Rank,
    text=learn-to-rank,
    description=A class of supervised machine learning approaches to develop adaptive ranking methods
}

\newglossaryentry{mean_average_precision}
{
    name=Mean Average Precision,
    text=mean average precision,
    description=An information retrieval ranking evaluation measure for capturing the mean of the average precision scores for each query
}

\newglossaryentry{normalized_discounted_cumulative_gain}
{
    name=Normalized Discounted Cumulative Gain,
    text=normalized discounted cumulative gain,
    description=An information retrieval ranking evaluation measure for capturing non-binary notions of relevance normalized by lengths of search results
}

\newglossaryentry{online_Evaluation}
{
    name=Online Evaluation,
    text=online evaluation,
    description={evaluation of a fully functioning system based on implicit measurement of real users’experiences of the system in a natural usage environment \cite{Hofmann2016}}
}

\newglossaryentry{precision}
{
    name=Precision,
    text=precision,
    description=An information retrieval ranking evaluation measure for the fraction of documents retrieved that are relevant (true positives) to a user's needs
}

\newglossaryentry{portals}
{
    name=Geospatial Data Portal,
    text=portals,
    description={Web catalogs that provide a platform for discovering, exploring, and retrieving geospatial data}
}

\newglossaryentry{recall}
{
    name=Recall,
    text=recall,
    description=An information retrieval ranking evaluation measure for capturing the fraction of all relevant (true positive) documents that are retrieved
}

\newglossaryentry{relevance}
{
    name=Relevance,
    text=relevance,
    description=The relation between user needs and what can be served by  the  entire  information  ecology \cite{Hjorland2010}
}

\newglossaryentry{Rocchio_classification}
{
    name=Rocchio Classification,
    text=Rocchio classification,
    description={A vector space classification where a query is based on the concept that most users have a general conception of what a relevant document is. Based on this assumption, a search query will be expanded by an arbitrary percentage of relevant and non-relevant documents as a means of increasing a search engine's recall, and possibly the precision as well \cite{Manning2008}}.
}

\newglossaryentry{support_vector_machines}
{
    name=Support Vector Machines,
    text=Support Vector Machines,
    description=A supervised machine learning classification model that separates classes using an optimized hyper plane
}

\newglossaryentry{tf_idf}
{
    name=Term Frequency–Inverse Document Frequency,
    text=term frequency–inverse document frequency,
    description=An information retrieval numerical statistic that indicates the importance of a term in a document in relation to all documents in a corpus. A term is considered highly important if it is frequent in a document but not frequent across documents in a corpus
}

\newglossaryentry{vector_space_model}
{
    name=Vector Space Model,
    text=Vector Space Model,
    description={An algebraic model for representing an entity by vector identifiers, such as a query by term vectors or a document by text metadata attributes.}
}

\newglossaryentry{Zipf_law}
{
    name=Zipf's Law,
    text=Zipf's Law,
    description={An empirical law that suggests text follows a Zipfian distribution, that is the rank of a word times its frequency in a corpus is approximately constant}
}