\chapter{Significance} \label{ch:[chapter 6 label]}

Understanding relevance in IR is notoriously difficult. However, top search engines are where they are today because of their continual evaluations of relevance and adjustments to ranking through techniques like learn-to-rank. This research is a first attempt at a sustainable evaluation process like this for GIR of geospatial datasets. To the best of my knowledge, this research is the first attempt at studying GIR of geospatial datasets. As mentioned previously, similar and more in depth studies have been conducted for other types of GIR systems like those that serve information for wayfinding or location based services (LBS). This research is also significant because it is a refinement on previous work deducing GR. Enumerating concepts that imply relevance and seeing how they relate to search behavior has not been studied in this context.

Computational implementation of GR remains relatively simple, only serving data that are "at" a place. This research is also significant because of its technical application. Most likely, my research will be directly applied to improving OpenData’s search systems by expanding their search criteria and facets. OpenData could be a model for improving other Esri search systems such as ArcGIS Online and Living Atlas. Lastly, this research is significant because of its implications for cognitive science. Specifically, the epistemological-view treats GIR as a task-solving process with user needs in mind. This research will serve as a foundation for future GIR system development based on user’s GINs. To the best of my knowledge, there has been little cognitive research on GINs besides needs in wayfinding and orientation tasks.

*[todo] change to: and social (broader impact of finding relevant geographic information)*

*add bits from extras document*