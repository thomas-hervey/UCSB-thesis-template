\chapter{Significance} \label{ch:[chapter 6 label]}

Understanding relevance in IR is notoriously difficult. However, top search engines are where they are today because of their continual evaluations of relevance and adjustments to ranking through techniques like learn-to-rank. This research is significant because, to the best of my knowledge, it is a first attempt at an extensive study and evaluation of GIR for geospatial datasets. It is significant because of the unique research datasets and the research context. As mentioned previously, similar and more in depth studies have been conducted for other types of GIR systems like those that serve information for wayfinding or location based services (LBS) \cite{DeSabbata2012}. This research is also significant because it is a refinement on previous work deducing GR. Enumerating concepts that imply relevance and seeing how they relate to search behavior has not been studied in this context.

Computational implementation of GR remains relatively simple, only serving data that are "at" a place. This research is also significant because of its technical application. It is significant to catalog concepts that users use when they search for geospatial data. It is also significant to use novel methods for indexing geospatial datasets to capture a richer notion of relevance. Most likely, my research will be directly applied to improving OpenData search systems by expanding their search criteria and facets. OpenData could be a model for improving other Esri search systems such as ArcGIS Online and Living Atlas. Lastly, this research is significant because of its broader implications for social and cognitive sciences. Specifically, the epistemological-view treats GIR as a task-solving process with user needs in mind. It gives a new look at how people behave in a specialized search environment, how they articulate their mental models, and how they ask about and refine the concepts that exist in the real world, not just a search system. This research will serve as a foundation for future GIR system development based on user’s GINs. To the best of my knowledge, there has been little cognitive research on GINs besides needs in wayfinding and orientation tasks.

Unlike other GIR studies, this one attempts to be comprehensive. In other words, the three different perspectives, system-view, user-view, and epistemological-view, capture almost all aspects of the GIR research process. This research starts by taking stock of search system capabilities, and then makes system improvements based on observations of user behavior. There are strengths and weaknesses to this approach. Each study is built off of the results from the previous one, making my research integrated and strong. Time and data limitations and certain assumptions about relevance are this research’s weaknesses. The following sections discuss these points in detail.

\section{Study Strengths}

This research uses well established IR methods for discovery (i.e., query log analysis), application (i.e., search system modification), and evaluation (i.e., system relevance measures and user relevance feedback). This is important because it makes my research easily reproducible and comparable. Also, the datasets in this research are large. This is a strength because large datasets reduce the risk of bias in search behavioral analysis. It is far less likely that common queries, refinements, or result clicks would happen by chance. Large datasets are also valuable because they could be used for training machine learning models, such as learn-to-rank, a popular method for incrementally improving a search ranking algorithm.

\section{Study Weaknesses}

This research suffers from a number of weaknesses as well. It could be argued that each of the five studies is not deep enough to be statistically significant. Although I plan to conduct significance testing in my fifth study on evaluation, the other study topics could certainly be explored further. Also, the datasets being used are rich, but not rich enough for some IR research standards. For example, in study two, I will not likely have access to search sessions, so I won’t be able to explore a specific information seeking activity in its entirety. Nor will I ask users to perform controlled search tasks or ask them why they performed specific system interactions in study five.

This research does not explore or employ semantic technology as much as it could. Search is often heavily (and rightfully so) about themes. While this research does not exclude theme, it focuses on geography (and how it covaries with theme). In the future, semantic networks like ConceptNet could help better integrate geographic concepts with thematic ones. Lastly, albeit a weakness of relevance in IR more broadly, this research cannot directly link user behavior to relevance. As mentioned in the research methods, user behavior is at best a proxy for relevance. But in my research, I will likely have to make additional assumptions, such as clicks and downloads indicating relevance since I will not have annotated relevance judgments prior to analysis.