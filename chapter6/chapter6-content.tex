\chapter{Significance} \label{ch:[chapter 6 label]}

Understanding relevance in IR is notoriously difficult. However, top search engines are where they are today because of their continual evaluations of relevance and adjustments to ranking through techniques like learn-to-rank. This research is significant because, to the best of my knowledge, it is a first attempt at an extensive study and evaluation of GIR for geospatial datasets. It is significant because of the unique research datasets and the research context. As mentioned previously, similar and more in depth studies have been conducted for other types of GIR systems like those that serve information for wayfinding or location based services (LBS) \cite{DeSabbata2012}. This research is also significant because it is a refinement on previous work deducing GR. Enumerating concepts that imply relevance and seeing how they relate to search behavior has not been studied in this context.

Computational implementation of GR remains relatively simple, only serving data that are "at" a place. This research is also significant because of its technical application. It is significant to catalog concepts that users use when they search for geospatial data. It is also significant to use novel methods for indexing geospatial datasets to capture a richer notion of relevance. Most likely, my research will be directly applied to improving OpenData search systems by expanding their search criteria and facets. OpenData could be a model for improving other Esri search systems such as ArcGIS Online and Living Atlas. Lastly, this research is significant because of its broader implications for social and cognitive sciences. Specifically, the epistemological-view treats GIR as a task-solving process with user needs in mind. It gives a new look at how people behave in a specialized search environment, how they articulate their mental models, and how they ask about and refine the concepts that exist in the real world, not just a search system. This research will serve as a foundation for future GIR system development based on user’s GINs. To the best of my knowledge, there has been little cognitive research on GINs besides needs in wayfinding and orientation tasks.